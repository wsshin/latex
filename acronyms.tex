\usepackage[single,mark-as-used=first,cite=first]{acroshin}  % mark as used only when the "first form" is used


% \NewDocumentCommand \acf { sm }
%   {
%     \group_begin:
%       \acro_defined:n { #2 }
%       \IfBooleanTF { #1 }
%         { \acro_use_if:n { * } }
%         { \acro_use_if:n {   } }
%       \acro_first:n { #2 }
%     \group_end:
%   }


% Some acronyms are used without definite and indefinite articles, whereas the articles are used in their long forms.  When some adjectives qualifies these terms, it becomes a problem.  For example, "the more commonly used uniaxial PML (UPML)" cannot be written as "the more commonly used \ac{upml}".
% Also, there are acronyms that include trailing words whose first characters do not appear in the acronyms.  (e.g., QMR for the quasi-minimal residual "method".)  When a family of these terms are used, it also becomes a problem.  For example, "the conjugate gradient (CG), biconjugate gradient (BiCG), and quasi-minimal residual (QMR) methods" cannot be written as "\ac{cg}, \ac{bicg}, and \ac{qmr}".
% Because there are too many problems in using \ac{...} only, use it sparingly, especially those with the above problems.  Use \acf{} when the term is used for the first time, and use \acs{} when it should be an acronym.
% Or, if I really want to use \ac{...}, then I should think about whether it will look good in all of the \acf{}, \acl{}, \acs{} forms.
\DeclareAcronym{fdfd}{
	short = FDFD,
	long = finite-difference frequency-domain
}
\DeclareAcronym{fdtd}{
	short = FDTD,
	long = finite-difference time-domain
}
\DeclareAcronym{fem}{
	short = FEM,
	long = finite element method
}
\DeclareAcronym{bem}{
	short = BEM,
	long = boundary element method
}
\DeclareAcronym{mom}{
	short = MoM,
	long = method of moments
}
\DeclareAcronym{pml}{
	short = PML,
	long = perfectly matched layer
}
\DeclareAcronym{scpml}{
	short = SC-\acs{pml},
	long = stretched-coordinate \acls{pml},
	short-plural = ,
	long-plural = 
}
\DeclareAcronym{upml}{
	short = U\acs{pml},
	long = uniaxial \acls{pml},
	short-plural = ,
	long-plural = 
}
\DeclareAcronym{cpml}{
	short = C\acs{pml},
	long = convolutional \acls{pml},
	short-plural = ,
	long-plural = 
}
\DeclareAcronym{spupml}{
	short = SP-\acs{upml},
	long = scale-factor-preconditioned \acls{upml},
	short-plural = ,
	long-plural = 
}
\DeclareAcronym{2d}{
	short = 2D,
	long = two-dimensional
}
\DeclareAcronym{3d}{
	short = 3D,
	long = three-dimensional
}
\DeclareAcronym{em}{
	short = EM,
	long = electromagnetic
}
\DeclareAcronym{cg}{
	short = CG,
	long = conjugate gradient
}
\DeclareAcronym{bicg}{
	short = BiCG,
	long = biconjugate gradient
}
\DeclareAcronym{qmr}{
	short = QMR,
	long = quasi-minimal residual
}
\DeclareAcronym{gmres}{
	short = GMRES,
	long = generalized minimal residual
}
\DeclareAcronym{mdm}{
	short = MDM,
	long = metal-dielectric-metal
}
\DeclareAcronym{gpu}{
	short = GPU,
	long = graphics processing unit
}
\DeclareAcronym{svd}{
	short = SVD,
	long = singular value decomposition
}
\DeclareAcronym{arpack}{
	short = ARPACK,
	long = Arnoldi Package
}
\DeclareAcronym{petsc}{
	short = PETSc,
	long = {portable, extensible toolkit for scientific computation}
}
\DeclareAcronym{pec}{
	short = PEC,
	long = perfect electric conductor
}
\DeclareAcronym{ic}{
	short = IC,
	long = integrated circuit
}
\DeclareAcronym{mpb}{
	short = MPB,
	long = MIT Photonic-Bands
}
\DeclareAcronym{te}{
	short = TE,
	long = transverse-electric
}
\DeclareAcronym{tm}{
	short = TM,
	long = transverse-magnetic
}
\DeclareAcronym{tem}{
	short = TEM,
	long = transverse-electromagnetic
}
