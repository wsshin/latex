\usepackage[single,mark-as-used=first,cite=first]{acro}  % mark-as-used only when the "first form" is used

% Some acronyms are used without definite and indefinite articles, whereas the articles are used in their long forms.  When some adjectives qualifies these terms, it becomes a problem.  For example, "the more commonly used uniaxial PML (UPML)" cannot be written as "the more commonly used \ac{upml}".
% Also, there are acronyms that include trailing words whose first characters do not appear in the acronyms.  (e.g., QMR for the quasi-minimal residual "method".)  When a family of these terms are used, it also becomes a problem.  For example, "the conjugate gradient (CG), biconjugate gradient (BiCG), and quasi-minimal residual (QMR) methods" cannot be written as "\ac{cg}, \ac{bicg}, and \ac{qmr}".
% Because there are too many problems in using \ac{...} only, use it sparingly, especially those with the above problems.  Use \acf{} when the term is used for the first time, and use \acs{} when it should be an acronym.
% Or, if I really want to use \ac{...}, then I should think about whether it will look good in all of the \acf{}, \acl{}, \acs{} forms.

% General
\DeclareAcronym{2d}{
	short = 2D,
	long = two-dimensional
}
\DeclareAcronym{3d}{
	short = 3D,
	long = three-dimensional
}
\DeclareAcronym{pi}{
	short = PI,
	long = Principal Investigator
}
\DeclareAcronym{bs}{
	short = B.S.,
	long = Bachelor of Science
}
\DeclareAcronym{ms}{
	short = M.S.,
	long = Master of Science
}
\DeclareAcronym{phd}{
	short = Ph.D.,
	long = Doctor of Philosophy
}

% EM
\DeclareAcronym{em}{
	short = EM,
	long = electromagnetic
}
\DeclareAcronym{mdm}{
	short = MDM,
	long = metal-dielectric-metal
}
\DeclareAcronym{te}{
	short = TE,
	long = transverse-electric
}
\DeclareAcronym{tm}{
	short = TM,
	long = transverse-magnetic
}
\DeclareAcronym{tem}{
	short = TEM,
	long = transverse-electromagnetic
}
\DeclareAcronym{pec}{
	short = PEC,
	long = perfect electric conductor
}

% Discetization methods
\DeclareAcronym{fdfd}{
	short = FDFD,
	long = finite-difference frequency-domain
}
\DeclareAcronym{fdtd}{
	short = FDTD,
	long = finite-difference time-domain
}
\DeclareAcronym{fem}{
	short = FEM,
	long = finite element method
}
\DeclareAcronym{fdm}{
	short = FDM,
	long = finite difference method
}
\DeclareAcronym{bem}{
	short = BEM,
	long = boundary element method
}
\DeclareAcronym{mom}{
	short = MoM,
	long = method of moments
}

% PMLs
\DeclareAcronym{pml}{
	short = PML,
	long = perfectly matched layer
}
\DeclareAcronym{scpml}{
	short = SC-\acs{pml},
	long = stretched-coordinate \acifused{pml}{\acs{pml}}{\acl{pml}},
	short-plural = ,
	long-plural = 
}
\DeclareAcronym{upml}{
	short = U\acs{pml},
	long = uniaxial \acifused{pml}{\acs{pml}}{\acl{pml}},
	short-plural = ,
	long-plural = 
}
\DeclareAcronym{cpml}{
	short = C\acs{pml},
	long = convolutional \acifused{pml}{\acs{pml}}{\acl{pml}},
	short-plural = ,
	long-plural = 
}
\DeclareAcronym{spupml}{
	short = SP-\acs{upml},
	long = scale-factor-preconditioned \acifused{upml}{\acs{upml}}{\acl{upml}},
	short-plural = ,
	long-plural = 
}

% Mathematics
\DeclareAcronym{cg}{
	short = CG,
	long = conjugate gradient
}
\DeclareAcronym{cgs}{
	short = CGS,
	long = conjugate gradient squared
}
\DeclareAcronym{bicg}{
	short = BiCG,
	long = biconjugate gradient
}
\DeclareAcronym{qmr}{
	short = QMR,
	long = quasi-minimal residual
}
\DeclareAcronym{gmres}{
	short = GMRES,
	long = generalized minimal residual
}
\DeclareAcronym{svd}{
	short = SVD,
	long = singular value decomposition
}
\DeclareAcronym{ilu}{
	short = ILU,
	long = incomplete LU
}
\DeclareAcronym{fft}{
	short = FFT,
	long = fast Fourier transform
}
\DeclareAcronym{pde}{
	short = PDE,
	long = partial differential equation
}

% Technologies
\DeclareAcronym{ic}{
	short = IC,
	long = integrated circuit
}
\DeclareAcronym{it}{
	short = IT,
	long = information technology
}
\DeclareAcronym{cad}{
	short = CAD,
	long = computer-aided design
}
\DeclareAcronym{flops}{
	short = FLOPS,
	long = floating-point operations per second
}
\DeclareAcronym{spm}{
	short = SPM,
	long = scanning probe microscope
}
\DeclareAcronym{gpu}{
	short = GPU,
	long = graphics processing unit
}
\DeclareAcronym{hpc}{
	short = HPC,
	long = high-performance computing
}

% Software
\DeclareAcronym{arpack}{
	short = ARPACK,
	long = Arnoldi Package
}
\DeclareAcronym{petsc}{
	short = PETSc,
	long = {portable, extensible toolkit for scientific computation}
}
\DeclareAcronym{fftw}{
	short = FFTW,
	long = Fastest Fourier Transform in the West
}
\DeclareAcronym{mpb}{
	short = MPB,
	long = MIT Photonic-Bands
}
\DeclareAcronym{meep}{
	short = MEEP,
	long = MIT Electromagnetic Equation Propagation
}
\DeclareAcronym{vc++}{
	short = VC++,
	long = Visual C++
}
\DeclareAcronym{j2se}{
	short = J2SE,
	long = {Java 2 Platform, Standard Edition}
}

% Organizations
\DeclareAcronym{tacc}{
	short = TACC,
	long = Texas Advanced Computing Center
}
\DeclareAcronym{sdsc}{
	short = SDSC,
	long = San Diego Supercomputer Center
}
\DeclareAcronym{psc}{
	short = PSC,
	long = Pittsburgh Supercomputing Center
}
\DeclareAcronym{ncsa}{
	short = NCSA,
	long = National Center for Supercomputing Applications
}
\DeclareAcronym{ams}{
	short = AMS,
	long = American Mathematical Society
}
\DeclareAcronym{siam}{
	short = SIAM,
	long = Society for Industrial and Applied Mathematics
}
\DeclareAcronym{arl}{
	short = ARL,
	long = Army Research Laboratory
}
\DeclareAcronym{afosr}{
	short = AFOSR,
	long = Air Force Office of Scientific Research
}
\DeclareAcronym{muri}{
	short = MURI,
	long = Multidisciplinary University Research Initiative
}
\DeclareAcronym{nsf}{
	short = NSF,
	long = National Science Foundation
}
\DeclareAcronym{dms}{
	short = DMS,
	long = Division of Mathematical Sciences
}
\DeclareAcronym{kmo}{
	short = KMO,
	long = Korean Mathematical Olympiad
}
\DeclareAcronym{kms}{
	short = KMS,
	long = Korean Mathematical Society
}
\DeclareAcronym{kosef}{
	short = KOSEF,
	long = Korea Science and Engineering Foundation
}

% Universities and Departments
\DeclareAcronym{ee}{
	short = EE,
	long = Electrical Engineering
}
\DeclareAcronym{cs}{
	short = CS,
	long = Computer Science
}
\DeclareAcronym{eecs}{
	short = EECS,
	long = \acl{ee} \& \acl{cs}
}
\DeclareAcronym{mit}{
	short = MIT,
	long = Massachusetts Institute of Technology
}
\DeclareAcronym{snu}{
	short = SNU,
	long = Seoul National University
}
\DeclareAcronym{uiuc}{
	short = UIUC,
	long = University of Illinois at Urbana-Champaign
}
\DeclareAcronym{lsu}{
	short = LSU,
	long = Louisiana State University
}
\DeclareAcronym{msr}{
	short = MSR,
	long = Microsoft Research
}

